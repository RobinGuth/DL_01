\documentclass[a4paper]{article}

\usepackage{listings}
\usepackage{graphicx}
\usepackage{float}
\usepackage[hidelinks]{hyperref}

%opening
\title{Deep Reinforcement Learning Beginners Tutorial \\Documentation}
\author{Julian Bernhart, Robin Guth}

\begin{document}
	
	\maketitle
	\tableofcontents
	
	\section{Introduction}
	
		 Deep Reinforcement Learning (also called RL) is a huge step towards the creation of an universal artificial intelligence. In 2013 a company, owned by Google, called ''Deep Mind'', was able to create an astonishing implementation of RL, which was capable to play retro games of the console ''Atari 2600''. In many cases the AI (Artificial Intelligence) was not only able to play the games successfully, but also exceeded human performances significantly. After these impressive results, it is definitly worth to take a closer look at Reinforcement Learning.
	
	legitimation:	 
		 Playing games is fun, but Reinforcement Learning has more to offer, as discussed in [3] . Apart from playing video games, there are use cases in many fields like robotics, traffic control or even personalized advertisement. While supervised and unsupervised learning are already used widely in production, reinforcement learning is still in development and further research is needed. As a fairly new topic, beginners often struggle to find a good starting point into the world of AI and specifically RL. Many tutorials are written for more advanced users, who already have a deeper unterstanding of machine learning. The ''Deep Reinforcement Learning Beginners Tutorial" will provide an easy-to-follow, hands-on beginners guide to RL. After the completion, we will be able to write our own algorithm to play some basic games for us.
		 
		 Before we head into the world of Reinforcement Learning, we will have to talk about software agents.
	
	goals:
		 ## Why game?
		 
		 ## notebooks
		 ## installation guide /notebooks -> Startschwierigkeiten -> guide
		 

	\section{Content}
	motivation	
	
	RL:
		agent controls
		concept etc.

	\section{Methods and Materials}
	
	keras etc
	jupyter lab
	open ai gym: 
		As we discussed before, Reinforcement Learning can be used to solve a range of different problems. Developing Machine Learning algorithms is often not easy to understand nor comprehensible especially for beginners. Furthermore, it is important to be able to compare the performance of different iterations of our algorithm, to be able to improve it.
		
		So bascially we need an environment, that we can use to test and train our RL agent, which fulfills the following requirements:
		
		repeatable test/training epochs
		finite set of inputs
		finite set of actions
		easy state representation
		easy to control agent
		deliver a score for a given state
		
		In practice, not all of these points will be fulfilled, but as this is a beginners guide, we will start with a simple environment. Luckily, many video games can be used as quite good environments for machine learning purposes. Many implementations of RL are tested with games as Benchmark and there are some good reasons for this. Developing a whole test environment would be labour intensive and would require dedicated work towards a useable simulator. Using an existing game is also easier to compare to human performance and therefore the evaluation of different algorithms is easier. Another important point is the size of possible inputs and actions. The AI replaces the human player. Depending on the game, the input for our agent is an image, like a human player would see it. The set of actions is a combination of different buttons, which can be pressed on a controller. Finally, games are fun and most people can relate to them. It is also easier to understand what we want to accomplish, because we can transfer aspects from our human play style to the behaviour of an AI. 

	
	\section{Results}
	critic/ expensions:
		using a framework
		policy learning / value learning -> both variants as cartpole agent
		policy gradient
	
	notebooks:
		theory, exercises
	installation guide
	
	outlook
	
	
	%\bibliography{bib} 
	%\bibliographystyle{ieeetr}
\end{document}
